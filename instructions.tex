To make a website hosted on IIS publicly accessible, you need to ensure that the server is properly configured to allow external access. Here are the key steps:
1. Assign a Public IP Address

Your server must have a public IP address, which can be directly accessed over the internet. This can be achieved by:

    Direct Public IP Assignment:
        If your server is directly connected to the internet, it may already have a public IP address. You can check this in the server's network settings.

    Using a Router with Port Forwarding:
        If your server is behind a router, you need to set up port forwarding. Forward incoming traffic on the router's public IP to the internal IP address of your server on the appropriate port (usually port 80 for HTTP and port 443 for HTTPS).

2. Domain Name Configuration

If you have a domain name, you can point it to your server's public IP address:

    DNS A Record:
        Go to your domain registrar's DNS management panel.
        Create an A record that points your domain or subdomain to the server's public IP address.

3. Configure IIS Site Binding

Ensure your IIS site is configured to respond to the appropriate IP address, hostname, and port:

    Open IIS Manager:
        You can open IIS Manager by typing inetmgr in the Run dialog or searching for it in the Start menu.

    Select Your Site:
        In the Connections pane, expand the server node and select the site you want to make public.

    Edit Bindings:
        In the Actions pane, click on Bindings.
        In the Site Bindings window, ensure there is a binding for the correct IP address, port, and hostname. If necessary, add a new binding with the public IP and the correct hostname.

4. Firewall Configuration

Ensure that your server's firewall allows incoming connections on the necessary ports:

    Windows Firewall:
        Open the Control Panel and navigate to System and Security > Windows Defender Firewall.
        Click on Advanced settings.
        In the left pane, click on Inbound Rules.
        If an inbound rule for IIS (or HTTP and HTTPS) does not exist, create one that allows incoming connections on ports 80 and 443.

    Network Firewall:
        If your server is behind an additional firewall (such as a hardware firewall or cloud-based security service), ensure it allows traffic on the relevant ports.

5. SSL/TLS Configuration (HTTPS)

For secure access, set up HTTPS by obtaining an SSL/TLS certificate:

    Obtain a Certificate:
        You can obtain a certificate from a trusted Certificate Authority (CA) or use a free CA like Let's Encrypt.

    Install the Certificate:
        Install the certificate in IIS Manager by going to Server Certificates.

    Configure HTTPS Binding:
        In the Bindings dialog for your site, add an HTTPS binding and select the installed certificate.

6. Testing and Verification

    Access the Site Externally:
        Try to access the site from an external network (not connected to the server's local network) using the public IP address or domain name.

    Check for Issues:
        Verify that the site loads correctly, and there are no issues with firewall rules, DNS settings, or SSL/TLS certificates.

7. Security Considerations

    Secure Your Server: Ensure your server is secure, with regular updates, strong passwords, and minimal exposed services.
    Use HTTPS: Always prefer HTTPS over HTTP to protect data in transit.
    Monitor Traffic: Use monitoring tools to keep track of traffic and detect any unauthorized access attempts.